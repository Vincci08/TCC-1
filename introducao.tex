\chapter{Introdução}\label{CAP:introducao}
%\thispagestyle{empty}
O consumo e o valor pago de energia elétrica residencial no Brasil tem aumentado de forma contínua, onde as condições climáticas desfavoráveis para a produção de energia tem tornado cada vez mais caro o consumo da mesma [1]. Com o tempo e avanço tecnológico, os aparelhos eletrodomésticos passaram a ser mais acessiveis a população, com isso outro fator que contribui bastante para o desperdício de energia, é a falha humana, tendo em vista que por descuido muitas vezes, equipamentos e aparelhos consomem energia de forma desnecessária, onde é comum encontrar em residências, luzes, tvs, ar-condicionado e etc, ligados sem de fato ter alguém utilizando.

Além disso atualmente é comum encontrar aparelhos eletrodomésticos que possuem a função de \textit{Stand-by}, onde o aparelho de fato não desliga e sim fica temporariamente em repouso, ou seja continua consumindo mesmo que e esteja "desligado". Apesar desta função consumir pouco, uma residência com vários equipamentos em \textit{Standy-by} pode aumentar consideravelmente o valor da fatura de energia no fim do mês, onde segundo o Instituto Akatun o consumo desses equipamentos podem ficar entre 15\% a 45\% do valor total da fatura [3].

Com isso, técnicas de monitoramento tem sido desenvolvidas para melhor controlar e/ou monitorar dispositivos e a internet tem sido a solução para diversos problemas, a mesma é utilizada para conexão de dispositivos, como por exemplo fornecer em tempo real as variáveis de monitoramento de um sensor [4] para gerenciamento eficiente de energia elétrica em uma residência.

O presente trabalho tem como objetivo criar um sistema IoT para monitoramento e controle do consumo de energia em residências, para evitar desperdícios de energia e fornecer qual o custo geral e específico gasto na residência, como também o controle de forma remota de taís dispositivos que estão sendo monitorados.


%\section{Motivação e objetivos}


 
%\section{Contribuicoes}




%\section{Producao cientifica}


%\section{Organizacao da tese}

%\noindent \textbf{Capitulo \ref{CAP2}}: 

%\noindent \textbf{Capitulo \ref{CAP3}}: 

%\noindent \textbf{Capitulo \ref{CAP4}}: 

%\noindent \textbf{Capitulo \ref{CAP5}}: 
