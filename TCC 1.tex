\documentclass[ruledheader,noindentfirst,anapcustomindent,abntfigtabnum,tocpage=plain]{abnt}
\usepackage{amsmath, amssymb, amsthm, verbatim, amsfonts, amstext}
%\usepackage[latin1]{inputenc}
\usepackage[brazilian]{babel}
\usepackage[utf8]{inputenc}
\usepackage[T1]{fontenc}
\usepackage{dropping}
\usepackage{graphicx}
\usepackage[hang,small,bf]{caption}
\usepackage[abnt-etal-list=0,abnt-etal-text=it,abnt-and-type=&,abnt-emphasize=bf,abnt-full-initials=yes,alf,bibjustif]{abntcite}
\usepackage{fancyhdr}
\usepackage{makeidx}
\usepackage[none]{hyphenat}
\usepackage{color}
\usepackage{subfig}
\usepackage{algorithms}
\usepackage{algorithmic}
\usepackage{mdwlist}
\usepackage{bm}
\usepackage[titletoc,title]{appendix}
\usepackage{ltxtable}
\usepackage{longtable}
\usepackage{supertabular}
\usepackage{indentfirst}
\usepackage{color}
\usepackage{icomma}

\sloppy


%
%Tradução do pacote Algorithm para portugues
%
\renewcommand{\algorithmicrequire}{\textbf{Entrada:}}
\renewcommand{\algorithmicensure}{\textbf{Saída:}}
\renewcommand{\algorithmicend}{\textbf{fim}}
\renewcommand{\algorithmicif}{\textbf{se}}
\renewcommand{\algorithmicthen}{\textbf{então}}
\renewcommand{\algorithmicelse}{\textbf{senão}}
\renewcommand{\algorithmicelsif}{\algorithmicelse \, \algorithmicif}
\renewcommand{\algorithmicendif}{\algorithmicend \, \algorithmicif}
\renewcommand{\algorithmicfor}{\textbf{para}}
\renewcommand{\algorithmicforall}{\textbf{para todo}}
\renewcommand{\algorithmicdo}{\textbf{fazer}}
\renewcommand{\algorithmicendfor}{\algorithmicend \, \algorithmicfor}
\renewcommand{\algorithmicwhile}{\textbf{enquanto}}
\renewcommand{\algorithmicendwhile}{\algorithmicend \, \algorithmicwhile}
\renewcommand{\algorithmicloop}{\textbf{laço}}
\renewcommand{\algorithmicendloop}{\algorithmicend \, \algorithmicloop}
\renewcommand{\algorithmicrepeat}{\textbf{repetir}}
\renewcommand{\algorithmicuntil}{\textbf{até}}
\renewcommand{\algorithmiccomment}[1]{\{#1\}}
\renewcommand{\listalgorithmname}{Lista de Algoritmos}
\floatname{algorithm}{Algoritmo}
%%%%%%%%%%%%%%%%%%%%%%%%%%%%%%%%%%%%%%%%%%%%%%%%%%%%%%%%%%%%%%%%%%%%%%%%%%%%%%%%%%%

\makeindex

%%%% O arquivo modelosCAP.tex possui as definições para ciação do estilo de capítulo (fonte de título, barras horizontais, etc.)
% ele não gera texto de saída, é um arquivo de configuração somente
%
\input{modelosCAP}
%%%%%%%%%%%%%%%%%%%%%%%%%%%%%%%%%%%%%%%%%%%%%%%FIM DO PREAMBULO%%%%%%%%%%%%%%%%%%%%%%%%%%%%%%%%%%%%%%%%%%%%%%%%%%%%%%%%%%%%%%%%%%


\begin{document}

%%%%% IMPORTANTE: ALTERA O TEXTO ENTRE ARIAL E TIMES NEW ROMAN (ALTERNAR OS COMENTÁRIOS)
%
%%%%%%%%%%%%%%%%%%%%%PARA UTILIZAR ARIAL%%%%%%%%%%%%%%%%%%%%%%%
%
\fontfamily{phv}                    %fonte Arial
\renewcommand{\rmdefault}{phv}      %
%
%%%%%%%%%%%%%%%%%%%%%PARA UTILIZAR TIMES%%%%%%%%%%%%%%%%%%%%%%%
%
%\fontfamily{ptm}               %fonte Times
%\renewcommand{\rmdefault}{ptm} %
%
%%%%%%%%%%%%%%%%%%%%%%%%%%%%%%%%%%%%%%%%%%%%%%%%%%%%%%%%%%%%%%%

%%%%%%%%%%%%%Arquivos .tex com os elementos pré-textuais
%
\thispagestyle{empty}

\vfill
 \begin{center}
    \begin{figure}[t]
     \centering
            \includegraphics[width=5cm]{figures/IF_logo.eps}\\[-0.1in]
     \end{figure}

    {\large\bfseries INSTITUTO FEDERAL DE EDUCAÇÃO, CIÊNCIA E TECNOLOGIA DO CEARA} \\
    {\large\bfseries PRÓ-REITORIA DE ENSINO} \\
    {\large\bfseries COORDENADORIA DE TELEMÁTICA DO CAMPUS MARACANAÚ}  \\ 
    {\large\bfseries BACHARELADO EM CIÊNCIA DA COMPUTAÇÃO}  \\ 

    \vspace*{1in}
    \begin{large} \bfseries VINICIUS FERREIRA DA SILVA \end{large}\\[0.4in]

    \vspace*{4cm}
    \noindent \\
    \large\bfseries{SISTEMA IOT PARA MONITORAMENTO E CONTROLE DO CONSUMO DE ENERGIA RESIDENCIAL} \\
    \vfill
    \large\bfseries{ MARACANAÚ \\ 2020}
\end{center}

\normalsize
\begin{titlepage}
\vfill
\begin{center}

    {\large VINICIUS FERREIRA DA SILVA\\}
    \vspace{2cm}
    {\Large \textsc{SISTEMA IOT PARA MONITORAMENTO E CONTROLE DO CONSUMO DE ENERGIA RESIDENCIAL}\\}
    \vspace{1cm}
    \hspace{.45\linewidth}
    \begin{minipage}{.50\linewidth}

            Trabalho de conclusão de curso I do Curso de Bacharelado em Ciência da Computação do Instituto Federal do Ceará - Campus Maracanaú, como requisito parcial para obtenção do grau de Bacharel em Ciência da Computação.

            \vspace{0.5 cm}

            Área de pesquisa: IOT

            \vspace{0.5 cm}

            Orientador:D.r SANDRO CÉSAR SILVEIRA JUCÁ
    
    \end{minipage}

    \vspace{2cm}
    \vfill
    {\large Maracanaú\\ 2020}
\end{center}

\end{titlepage}
\include{folhadeaprovacao}
\include{dedicatoria}
\include{conclusoes}
\include{agradecimentos}
\include{epigrafe}
\pagestyle{plain}%%%%% Utilizar ESTILO PLAIN AQUI%%%%%%%
\chapter*{Resumo}

\noindent resumo
\chapter*{Abstract}
This work aims to create a system aimed at the management and control over electrical expenses in a residence through monitoring, using the concept of IoT. The proposed system monitors in real time the energy consumption in a home, as well as controls sockets and lamps, using information from current sensors. For this, the nodeMCU microcontroller is used to connect to the internet and use the IoT concept.


\noindent 



%%%Comandos para criação automática das listas
%
\tableofcontents
%\listoffigures
%\listoftables

%%%Comandos para criar outras listas não suportadas pelo pacote ABNTex%%%
%
%\pretextualchapter{Lista de Símbolos}
%\input{nomenclatura}
\newpage

%\pretextualchapter{Lista de Abreviacoes}
%\begin{basedescript}{\desclabelstyle{\pushlabel}\desclabelwidth{6em}}
\item[{fdp}] Função densidade de probabilidade%
\item[{fda}] Função de distribuição acumulada%
\item[{EMQ}] Erro médio quadrático%
\end{basedescript}
%\newpage
%%%%%%%%%%%%%%%%%%%%%%%%%%%%%%%%%%%%%%%%%%%%%%%%%%%%%%%%%%%%%%%%%%%%

%Capítulos passam a ter páginas numeradas
%
\pagestyle{fancy}

%resseta os contadores de capítulo e seção
%
\renewcommand{\chaptermark}[1]{\markboth{#1}{}}
\renewcommand{\sectionmark}[1]{\markright{\thesection\ #1}}

%%%%%%%%%%%%%%NÃO LEMBRO O QUE FAZ, APARENTEMENTE NADA, TESTAR DEPOIS
%\fancyhf{}%
%\fancyhead[RO,LE]{\large\slshape\thepage}%
%\fancyhead[CE]{\large\slshape\leftmark}%
%\fancyhead[CO]{\large\slshape\rightmark}%


%%% Outros arquivos .tex. É acoselhável utilizar vários arquivos, pelo menos um por capítulo
\chapter{Introdução}\label{CAP:introducao}
%\thispagestyle{empty}
O consumo e o valor pago de energia elétrica residencial no Brasil tem aumentado de forma contínua, onde as condições climáticas desfavoráveis para a produção de energia tem tornado cada vez mais caro o consumo da mesma [1]. Com o tempo e avanço tecnológico, os aparelhos eletrodomésticos passaram a ser mais acessiveis a população, com isso outro fator que contribui bastante para o desperdício de energia, é a falha humana, tendo em vista que por descuido muitas vezes, equipamentos e aparelhos consomem energia de forma desnecessária, onde é comum encontrar em residências, luzes, tvs, ar-condicionado e etc, ligados sem de fato ter alguém utilizando.

Além disso atualmente é comum encontrar aparelhos eletrodomésticos que possuem a função de \textit{Stand-by}, onde o aparelho de fato não desliga e sim fica temporariamente em repouso, ou seja continua consumindo mesmo que e esteja "desligado". Apesar desta função consumir pouco, uma residência com vários equipamentos em \textit{Standy-by} pode aumentar consideravelmente o valor da fatura de energia no fim do mês, onde segundo o Instituto Akatun o consumo desses equipamentos podem ficar entre 15\% a 45\% do valor total da fatura [3].

Com isso, técnicas de monitoramento tem sido desenvolvidas para melhor controlar e/ou monitorar dispositivos e a internet tem sido a solução para diversos problemas, a mesma é utilizada para conexão de dispositivos, como por exemplo fornecer em tempo real as variáveis de monitoramento de um sensor [4] para gerenciamento eficiente de energia elétrica em uma residência.

O presente trabalho tem como objetivo criar um sistema IoT para monitoramento e controle do consumo de energia em residências, para evitar desperdícios de energia e fornecer qual o custo geral e específico gasto na residência, como também o controle de forma remota de taís dispositivos que estão sendo monitorados.


%\section{Motivação e objetivos}


 
%\section{Contribuicoes}




%\section{Producao cientifica}


%\section{Organizacao da tese}

%\noindent \textbf{Capitulo \ref{CAP2}}: 

%\noindent \textbf{Capitulo \ref{CAP3}}: 

%\noindent \textbf{Capitulo \ref{CAP4}}: 

%\noindent \textbf{Capitulo \ref{CAP5}}: 

\include{cap2}
\include{cap3}
\include{cap4}
\include{cap5}


%%%% Estilo de citação ABNT e arquivo de bibitens (mybibliography.bib)
\bibliographystyle{abnt-alf}
\bibliography{mybibliography}

%|\apendice
%\include{appendices}

\chapter{Conclusões}\label{CAP:introducao}
\newpage

\chapter{Referências bibliográficas}\label{CAP:introducao}

\noindent \textbf{[1]} Bento, D. B., Fusinato, D. F., & JJ, A. C. (2018). Análise Da Eficiência Energética de Servidores Utilizando Soluções IoT e Ferramenta de Monitoramento. Teste2, Public Knowledge Project (PKP).

\noindent \textbf{[2]} Freitas, A A. D "Sistema IOT de monitoramento e gerenciamento de equipamentos elétricos em uma sala de estudos.", 2019, Trabalho de conclusão de curso, UFERSA.

\noindent \textbf{[3]} Silva, F. D. D. S. “Desenvolvimento de um protótipo IOT para comandar e monitorar tomadas”. 2019, Trabalho de conclusão de curso, UFERSA.

\noindent \textbf{[4]} Silva, V. F., Jucá, S. C. S., Pereira, R. I. S., Alves, Á. C., & Coutinho, J. P. “Free and customizable web application for Internet of Things devices monitoring”. 2019, Principia, p. 226 – 236.

\noindent \textbf{[5]} Fréu, R., Rebelatto, T. “Proposta de um sistema de controle e monitoramento de tomadas e interruptores”. 2018, Simpósio de Ciência, Inovação e Tecnologia, 66.
\newpage

\end{document}