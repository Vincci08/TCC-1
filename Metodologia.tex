\chapter{Metodologia}\label{CAP4}

O trabalho proposto visa inicialmente o desenvolvimento de um protótipo afim de testar a eficácia do sistema antes de aplicalo em um cenário real, em seguida o sistema será aplicado em tomadas reais para ser verificado a eficiência no controle do consumo de energia em uma resiência.

A contrução do protótipo é dividida em três partes, primeiro a definição do circuito, depois a montagem do protótipo e por fim a programação. Para o protótipo será utilizado uma tomada, dois módulos relés, um Esp8266 e um sensor de corrente. O sistema final vai partir da utilização desses mesmos componentes e circuito para cada tomada, para gerenciamento individual e conjunta das mesmas em uma residência.

O sistema final contará com um aplicativo para o gerenciamento das tomadas, onde através do mesmo será possível obter os dados de consumo e assim trazer o valor do consumo em tempo real.

Na figura a seguir temos um cronograma para o desenvolvimente do protótipo e do sistema por completo, como também a finalização do trabalho escrito.

\begin{figure} [H]
    \centering
    \includegraphics[scale=.93]{Figuras/Cronograma.png}
    \caption{Cronograma.}
    \label{fig:my_label}
\end{figure}
