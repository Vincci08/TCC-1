\chapter{Revisão Bibliográfica}\label{CAP3}

Atualmente há a preocupação em saber o quanto é consumido de energia elétrica em determinados ambientes, como um Datacenter por exemplo, onde analisar o consumo dos equipamentos torna-se uma importante ferramenta para os administradores visando monitorar o consumo e até mesmo o desempenho dos servidores em determinadas operações. Em [1] uma ferramenta de monitoramento é utilizada (Zabbix) junto aos conceitos e aplicações em IoT para a análise da eficiência energética em servidores nos datacenters. A análise visa quantificar o consumo de energia e o valor a ser pago pelo cosumo dos servidores. Ao final do trabalho é visto que é melhor utilizar um servidor operando a 99 \% do que dividir essas operaçõe para serem executadas para três servidores, pois o consumo de energia é menor.

Com o aumento do consumo de enerfia elétrica tem-se buscado formas eficientes de economizar energia com isso em [2] foi criado um dispositivo baseado em IoT para monitorar variáveis de temperatura, intensidade da luz e presença, em uma sala, com o objetivo inicial de apenas monitorar o consumo para assim desperdiçar o que era gasto sem necessidade, como por exemplo deixar lâmpadas e arcondicionados ligados sem de fator ter alguem na sala os utilizando. Com o dispositivo, ainda é possivel acioar os equipamentos ou desligalos, de forma altomatica ou em tempo real de forma online.

Em [3] é criado um protótipo capaz de obter estimativas de grandezas de uma tomada de forma remota, por meio da internet. O prototivo visa diminuir o consumo de energia em tomadas elétricas por equipamentos que desperdiçam energia por estarem no modo Stand-by, onde por meio do mesmo é possivel desligar as tomadas ao detectar uma corrente baixa em uma tomada principal. Para o protótipo, o autor utilizou um NodeMCU e uma platarforma parafazer toda a comunicação entre os dispositivos chamada Adafruit.

Em [4] é criado um aplicativo gratuito e personalizável para IoT, chamado Wireless Monitor. O aplicativo foi criado com o obhjetivo de fornecer uma solução para aquisição de dados coletados por sensores e o monitoramnto desses dados de forma segura, online e em tempo real. O aplicativo possui um sistema de plugins tornando o mesmo em aplicativo extensivo, pois o usuario pode utilizar vários sensores e os mesmo podem ser diferentes, sendo assim, um aplicativo customizavel para realizar monitoramentos de sensores.

Em [5] é dado a proposta da criação de ums sistema para controle e monitoramento de tomadas eletricas e interruptores, onde o mesmo possui três subsistemas. O primeiro serve para o controle dos interruptores, o segundo para o controle e monitoramento das tomadas e o terceiro é uma central de controle e armazenamento dos dados obtidos por meio dos sensores de corrente, onde os mesmo são armazenados em um Data Logger. Para o sistema foi utilizado o microcontrolador ESP8266.

O sistema proposto neste trabalho visa o monitoramento do consumo elétrico em residências, propondo ter em mãos e em tempo real o valor que está sendo consumido em um instante ou em um periodo, com um mês por exemplo, com isso é possivel ter um melhor controle dos gastos e evitar surpresas no final do mês com o valor da conta de energia. Além disso será possivel controlar e desligar tomadas e interruptores de forma remota.