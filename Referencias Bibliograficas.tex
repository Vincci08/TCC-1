\chapter{Referências bibliográficas}\label{CAP:introducao}

\noindent \textbf{[1]} Bento, D. B., Fusinato, D. F., & JJ, A. C. (2018). Análise Da Eficiência Energética de Servidores Utilizando Soluções IoT e Ferramenta de Monitoramento. Teste2, Public Knowledge Project (PKP).

\noindent \textbf{[2]} Freitas, A A. D "Sistema IOT de monitoramento e gerenciamento de equipamentos elétricos em uma sala de estudos.", 2019, Trabalho de conclusão de curso, UFERSA.

\noindent \textbf{[3]} Silva, F. D. D. S. “Desenvolvimento de um protótipo IOT para comandar e monitorar tomadas”. 2019, Trabalho de conclusão de curso, UFERSA.

\noindent \textbf{[4]} Silva, V. F., Jucá, S. C. S., Pereira, R. I. S., Alves, Á. C., & Coutinho, J. P. “Free and customizable web application for Internet of Things devices monitoring”. 2019, Principia, p. 226 – 236.

\noindent \textbf{[5]} Fréu, R., Rebelatto, T. “Proposta de um sistema de controle e monitoramento de tomadas e interruptores”. 2018, Simpósio de Ciência, Inovação e Tecnologia, 66.
\newpage